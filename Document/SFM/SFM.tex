% Template for ICASSP-2010 paper; to be used with:
%          mlspconf.sty  - ICASSP/ICIP LaTeX style file adapted for MLSP, and
%          IEEEbib.bst - IEEE bibliography style file.
% --------------------------------------------------------------------------
\documentclass{article}
\bibliographystyle{plain}

% use Times
\usepackage{times}
% For figures
\usepackage{graphicx} % more modern
%\usepackage{epsfig} % less modern
\usepackage{subfigure} 
% For algorithms
\usepackage{algorithm}
\usepackage{algorithmic}

% As of 2011, we use the hyperref package to produce hyperlinks in the
% resulting PDF.  If this breaks your system, please commend out the
% following usepackage line and replace \usepackage{icml2014} with
% \usepackage[nohyperref]{icml2014} above.
\usepackage{hyperref}

% Packages hyperref and algorithmic misbehave sometimes.  We can fix
% this with the following command.
\newcommand{\theHalgorithm}{\arabic{algorithm}}

\usepackage{color}
\usepackage{amsthm}
\usepackage{amsmath}
\usepackage{amsfonts}

\newtheorem{theo}{Theorem}
\newtheorem{prop}[theo]{Proposition}
\newtheorem{lemm}[theo]{Lemma}
\newtheorem{prof}[theo]{Proof}
\renewcommand{\theprof}{}

  
\DeclareMathOperator{\rch}{RCH_\eta}
\DeclareMathOperator{\ch}{CH}
\DeclareMathOperator{\argmin}{argmin}
\DeclareMathOperator{\argmax}{argmax}
\newcommand{\re}{\mathbb{R}}
\newcommand{\na}{\mathbb{N}}
\newcommand{\rand}{{\rm Random}}
\newcommand{\randre}{{\rm RandomReal}}
\newcommand{\red}{\mathbb{R}^d}
\newcommand{\bi}{\left\lbrace -1,1 \right\rbrace}
\newcommand{\m}{{\rm minimize}}
\newcommand{\M}{{\rm maximize}}
\newcommand{\st}{{\rm subject~ to}}
\newcommand{\ms}{R_\ominus}
\newcommand{\me}{E_\ominus }
\newcommand{\aff}{{\rm aff} }


%\address{
\begin{document}
%\fontsize{9pt}{10.8pt}\selectfont
\fontsize{10pt}{12pt}\selectfont
%\ninept
%

\title{How to Use SFM }
\author{onigiri}
\date{\today}


\maketitle
\newpage


\tableofcontents

\newpage

\section{Options}
We need to set some options in order to execute SFM.exe.

\subsection{Oracle}
We can set a submodular function oracle by "-o" option.
\begin{itemize}
\item 0: Undirected Cut
\item 1: Directed Cut
\item 2: Connected Detachment
\item 3: Facility Location
\item 4: Graphic Matroid
\item 5: Set Cover
\item 6: Non Positive Symmetric Matrix Summation
\item 7: Binary Matroid
\end{itemize}

See "Document\\DataCreator\\DataCreator.pdf" for details.

\subsection{Submodular Functions}
We can set an algorithm for submodular function minimization by "-a" option.
\begin{itemize}
\item 0: Iwata-Fleischer-Fujishige weakly polynomial time algorithm
\item 1: Iwata-Fleischer-Fujishige Strongly polynomial time algorithm
\item 2: Schrijver's algorithm
\item 3: Iwata's hybrid weakly polynomial time algorithm
\item 4: Iwata's hybrid strongly polynomial time algorithm
\item 5: Orlin's algorithm
\item 6: Iwata-Orlin weakly polynomial time algorithm
\item 7: Iwata-Orlin strongly polynomial time algorithm
\item 8: Fujishige-Wolfe algorithm
\end{itemize}

\subsection{Input File}
We can set the file for the input data by "-f" option.
See "Document\\DataCreator\\DataCreator.pdf" for the format of input data.

\subsection{Output File}
We can set the file for writing output by "-r" option.


\subsection{Numerical Error Tolerance}
We can set the numerical error tolerance by "-e" option.
We do not need to set this value and in that case,
the tolerance is $10^{-10}$.
Note that if we use severe error tolerance or non-integer submodular function,
it is possible for the program to take huge time.







\end{document} 
